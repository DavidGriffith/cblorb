\def\weavesectionsss#1#2#3#4#5{\vfill\eject\noindent{\titlefont #2}\mark{#3}\vskip 0.5in\par\noindent\titlepage}
\def\weavesectionss#1#2#3#4#5{\vfill\eject\noindent{\stitlefont #2\hfill #5}\mark{#3}\bigskip\par\noindent\titlepage}
\def\weavesections#1#2#3#4#5{\bigskip\goodbreak\noindent{\bf #4#1. #2.}\mark{#3}\quad}
\def\weavesection#1#2#3#4#5{\bigskip\goodbreak\noindent{\bf #4#1.}\mark{#3}\quad}
\def\nsweavesections#1#2#3#4#5{\noindent{\bf #4#1. #2.}\mark{#3}\quad}
\def\nsweavesection#1#2#3#4#5{\noindent{\bf #4#1.}\mark{#3}\quad}

\raggedbottom

\nopagenumbers
\newif\iftitle
\def\titlepage{\global\titletrue}
\headline={\iftitle\hfil\global\titlefalse\else\tenrm\hfil{\sourcefont \firstmark\qquad\the\pageno}\fi}

\titlepage

\def\quotesource#1{{\narrower\smallskip\noindent{\sourcefont #1}\smallskip}}
\def\cast#1#2{{#1}$\;\rightarrow\;${#2}}
\def\comp#1#2{{#1}$\;\sim\;${#2}}
\def\al{$\alpha$}
\def\be{$\beta$}
\def\nonterminal#1{$\langle${\it #1}$\rangle$}
\def\fixed#1{{\bf #1}}
\def\from{\par\qquad\quad$\leftarrow$\quad}
\def\continuation{\par\qquad\quad\phantom{$\leftarrow$}\quad}

\font\tentex=cmtex10
\font\sourcefont=cmss10
\font\xreffont=cmss9
\font\usagefont=cmss8

\font\inchhigh=cminch
\font\sinchhigh=cminch at 60pt
\font\titlefont=cmss10 at 24pt
\font\stitlefont=cmss10 at 20pt
\font\sstitlefont=cmss10 at 20pt

\font\ninerm=cmr9
\font\eightrm=cmr8
\font\sixrm=cmr6

\font\ninei=cmmi9
\font\eighti=cmmi8
\font\sixi=cmmi6
\skewchar\ninei='177 \skewchar\eighti='177 \skewchar\sixi='177

\font\ninesy=cmsy9
\font\eightsy=cmsy8
\font\sixsy=cmsy6
\skewchar\ninesy='60 \skewchar\eightsy='60 \skewchar\sixsy='60

\font\sevenss=cmss10 at 7pt
\font\eightss=cmssq8

\font\eightssi=cmssqi8

\font\ninebf=cmbx9
\font\eightbf=cmbx8
\font\sixbf=cmbx6

\font\ninett=cmtt9
\font\eighttt=cmtt8

\hyphenchar\tentt=-1 % inhibit hyphenation in typewriter type
\hyphenchar\ninett=-1
\hyphenchar\eighttt=-1

\font\ninesl=cmsl9
\font\eightsl=cmsl8

\font\nineit=cmti9
\font\eightit=cmti8

\font\tenu=cmu10 % unslanted text italic
\font\magnifiedfiverm=cmr5 at 10pt
\font\manual=manfnt % font used for the METAFONT logo, etc.
\font\cmman=cmman % font used for miscellaneous Computer Modern variations

\newskip\ttglue
\def\tenpoint{\def\rm{\fam0\tenrm}%
  \textfont0=\tenrm \scriptfont0=\sevenrm \scriptscriptfont0=\fiverm
  \textfont1=\teni \scriptfont1=\seveni \scriptscriptfont1=\fivei
  \textfont2=\tensy \scriptfont2=\sevensy \scriptscriptfont2=\fivesy
  \textfont3=\tenex \scriptfont3=\tenex \scriptscriptfont3=\tenex
  \def\it{\fam\itfam\tenit}%
  \textfont\itfam=\tenit
  \def\sl{\fam\slfam\tensl}%
  \textfont\slfam=\tensl
  \def\bf{\fam\bffam\tenbf}%
  \textfont\bffam=\tenbf \scriptfont\bffam=\sevenbf
   \scriptscriptfont\bffam=\fivebf
  \def\tt{\fam\ttfam\tentt}%
  \textfont\ttfam=\tentt
  \tt \ttglue=.5em plus.25em minus.15em
  \normalbaselineskip=12pt
  \def\MF{{\manual META}\-{\manual FONT}}%
  \let\sc=\eightrm
  \let\big=\tenbig
  \setbox\strutbox=\hbox{\vrule height8.5pt depth3.5pt width\z@}%
  \normalbaselines\rm}

\def\ninepoint{\def\rm{\fam0\ninerm}%
  \textfont0=\ninerm \scriptfont0=\sixrm \scriptscriptfont0=\fiverm
  \textfont1=\ninei \scriptfont1=\sixi \scriptscriptfont1=\fivei
  \textfont2=\ninesy \scriptfont2=\sixsy \scriptscriptfont2=\fivesy
  \textfont3=\tenex \scriptfont3=\tenex \scriptscriptfont3=\tenex
  \def\it{\fam\itfam\nineit}%
  \textfont\itfam=\nineit
  \def\sl{\fam\slfam\ninesl}%
  \textfont\slfam=\ninesl
  \def\bf{\fam\bffam\ninebf}%
  \textfont\bffam=\ninebf \scriptfont\bffam=\sixbf
   \scriptscriptfont\bffam=\fivebf
  \def\tt{\fam\ttfam\ninett}%
  \textfont\ttfam=\ninett
  \tt \ttglue=.5em plus.25em minus.15em
  \normalbaselineskip=11pt
  \def\MF{{\manual hijk}\-{\manual lmnj}}%
  \let\sc=\sevenrm
  \let\big=\ninebig
  \setbox\strutbox=\hbox{\vrule height8pt depth3pt width\z@}%
  \normalbaselines\rm}

\def\ttninepoint{\def\rm{\fam0\ninerm}%
  \textfont0=\ninerm \scriptfont0=\sixrm \scriptscriptfont0=\fiverm
  \textfont1=\ninei \scriptfont1=\sixi \scriptscriptfont1=\fivei
  \textfont2=\ninesy \scriptfont2=\sixsy \scriptscriptfont2=\fivesy
  \textfont3=\tenex \scriptfont3=\tenex \scriptscriptfont3=\tenex
  \def\it{\fam\itfam\nineit}%
  \textfont\itfam=\nineit
  \def\sl{\fam\slfam\ninesl}%
  \textfont\slfam=\ninesl
  \def\bf{\fam\bffam\ninebf}%
  \textfont\bffam=\ninebf \scriptfont\bffam=\sixbf
   \scriptscriptfont\bffam=\fivebf
  \def\tt{\fam\ttfam\ninett}%
  \textfont\ttfam=\ninett
  \normalbaselineskip=11pt
  \normalbaselines\rm}

\def\eightpoint{\def\rm{\fam0\eightrm}%
  \textfont0=\eightrm \scriptfont0=\sixrm \scriptscriptfont0=\fiverm
  \textfont1=\eighti \scriptfont1=\sixi \scriptscriptfont1=\fivei
  \textfont2=\eightsy \scriptfont2=\sixsy \scriptscriptfont2=\fivesy
  \textfont3=\tenex \scriptfont3=\tenex \scriptscriptfont3=\tenex
  \def\it{\fam\itfam\eightit}%
  \textfont\itfam=\eightit
  \def\sl{\fam\slfam\eightsl}%
  \textfont\slfam=\eightsl
  \def\bf{\fam\bffam\eightbf}%
  \textfont\bffam=\eightbf \scriptfont\bffam=\sixbf
   \scriptscriptfont\bffam=\fivebf
  \def\tt{\fam\ttfam\eighttt}%
  \textfont\ttfam=\eighttt
  \tt \ttglue=.5em plus.25em minus.15em
  \normalbaselineskip=9pt
  \def\MF{{\manual opqr}\-{\manual stuq}}%
  \let\sc=\sixrm
  \let\big=\eightbig
  \setbox\strutbox=\hbox{\vrule height7pt depth2pt width\z@}%
  \normalbaselines\rm}

\def\tenmath{\tenpoint\fam-1 } % use after $ in ninepoint sections
\def\tenbig#1{{\hbox{$\left#1\vbox to8.5pt{}\right.\n@space$}}}
\def\ninebig#1{{\hbox{$\textfont0=\tenrm\textfont2=\tensy
  \left#1\vbox to7.25pt{}\right.\n@space$}}}
\def\eightbig#1{{\hbox{$\textfont0=\ninerm\textfont2=\ninesy
  \left#1\vbox to6.5pt{}\right.\n@space$}}}

\catcode`@=11 % borrow the private macros of PLAIN (with care)

\def\|{\leavevmode\hbox{\tt\char`\|}} % vertical line

% macros for verbatim scanning
\chardef\other=12
\def\ttverbatim{\begingroup\ttninepoint\frenchspacing
  \catcode`\\=\other
  \catcode`\{=\other
  \catcode`\}=\other
  \catcode`\$=\other
  \catcode`\&=\other
  \catcode`\#=\other
  \catcode`\%=\other
  \catcode`\~=\other
  \catcode`\_=\other
  \catcode`\^=\other
  \obeyspaces \obeylines \tt}

\outer\def\begintt{$$\let\par=\endgraf \ttverbatim \parskip=\z@
  \catcode`\|=0 \rightskip-5pc \ttfinish}
{\catcode`\|=0 |catcode`|\=\other % | is temporary escape character
  |obeylines % end of line is active
  |gdef|ttfinish#1^^M#2\endtt{#1|vbox{#2}|endgroup|tenpoint$$}}

\catcode`\|=\active
{\obeylines \gdef|{\ttverbatim \spaceskip\ttglue \let^^M=\  \let|=\endgroup}}

% macros for syntax rules (again, not in Appendix E)
\def\[#1]{\silenttrue\xref|#1|\thinspace{\tt#1}\thinspace} % keyword in syntax
\def\beginsyntax{\endgraf\nobreak\medskip
  \begingroup \catcode`<=13 \catcode`[=13
  \let\par=\endsyntaxline \obeylines}
\def\endsyntaxline{\futurelet\next\syntaxswitch}
\def\syntaxswitch{\ifx\next\<\let\next=\syntaxrule
  \else\ifx\next\endsyntax\let\next=\endgroup
  \else\let\next=\continuerule\fi\fi \next}
\def\continuerule{\hfil\break\indent\qquad}
\def\endsyntax{\medbreak\noindent}
{\catcode`<=13 \catcode`[=13
  \global\let<=\< \global\let[=\[
  \gdef\syntaxrule<#1>{\endgraf\indent\silentfalse\xref\<#1>}}
\def\is{\ $\longrightarrow$ }
\def\alt{\ $\vert$ }

% macros to demarcate lines quoted from TeX source files
\def\beginlines{\par\begingroup\nobreak\medskip\parindent\z@ \obeylines
\nobreak \everypar{\strut}}
\def\endlines{\endgroup\medbreak\noindent}
